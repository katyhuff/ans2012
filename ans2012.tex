%        File: ans2012.tex
%     Created: Wednesday, January 25, 2012

\documentclass{anstrans}
%% To use the glossaries acronym package, you'll need to define any acronyms you intend to 
%% use. You can define acronyms with \newacronym{label}[acronym]{written out form}
%% To refer to them in the text use \gls{label}
\usepackage[acronym,toc]{glossaries}
\newacronym{MIT}{MIT}{the Massachusettes Institute of Technology}
\newacronym{UW}{UW}{University of Wisconsin}
\newacronym{US}{US}{United States}
\newacronym{SNF}{SNF}{spent nuclear fuel}
\newacronym{FEHM}{FEHM}{Finite Element Heat and Mass Transfer}
\newacronym{DOE}{DOE}{Department of Energy}
\newacronym{GENIUSv2}{GENIUS}{Global Evaluation of Nuclear Infrastructure Utilization Scenarios, Version 2}
\newacronym{CNERG}{CNERG}{Computational Nuclear Engineering Research Group}
\newacronym{GDSM}{GDSM}{Generic Disposal System Model}
\newacronym{GPAM}{GPAM}{Generic Performance Asessment Model}
\newacronym{FEPs}{FEPs}{Features, Events, and Processes}
\newacronym{EBS}{EBS}{Engineered Barrier System}
\newacronym{EDZ}{EDZ}{Excavation Disturbed Zone}
\newacronym{YMR}{YMR}{Yucca Mountain Repository Site}
\newacronym{EPA}{EPA}{Environmental Protection Agency}
\newacronym{PEI}{PEI}{Peak Environmental Impact}
\newacronym{VISION}{VISION}{the Verifiable Fuel Cycle Simulation Model}
\newacronym{NUWASTE}{NUWASTE}{Nuclear Waste Assessment System for Technical Evaluation}
\newacronym{NWTRB}{NWTRB}{Nuclear Waste Technical Review Board}
\newacronym{OCRWM}{OCRWM}{Office of Civillian Radioactive Waste Management}
\newacronym{UFD}{UFD}{Used Fuel Disposition}
\newacronym{DYMOND}{DYMOND}{Dynamic Model of Nuclear Development }
\newacronym{DANESS}{DANESS}{Dynamic Analysis of Nuclear Energy System Strategies}
\newacronym{CAFCA}{CAFCA}{ Code for Advanced Fuel Cycles Assessment }
\newacronym{ORION}{ORION}{O..}
\newacronym{NFCSim}{NFCSim}{Nuclear Fuel Cycle Simulator}
\newacronym{COSI}{COSI}{Commelini-Sicard}
\newacronym{FCT}{FCT}{Fuel Cycle Technology}
\newacronym{SWF}{SWF}{Separations and Waste Forms}
\newacronym{FCO}{FCO}{Fuel Cycle Options}
\newacronym{RDD}{RD\&D}{Research Development and Design}
\newacronym{WIPP}{WIPP}{Waste Isolation Pilot Plant}
\newacronym{ANDRA}{ANDRA}{Agence Nationale pour la gestion des D\'echets RAdioactifs, the French National Agency for Radioactive Waste Management}
\newacronym{TSM}{TSM}{Total System Model}
\newacronym{LANL}{LANL}{Los Alamos National Laboratory}
\newacronym{INL}{INL}{Idaho National Laboratory}
\newacronym{ANL}{ANL}{Argonne National Laboratory}
\newacronym{SNL}{SNL}{Sandia National Laboratory}
\newacronym{LBNL}{LBNL}{Lawrence Berkeley National Laboratory}
\newacronym{LLNL}{LLNL}{Lawrence Livermore National Laboratory}
\newacronym{NAGRA}{NAGRA}{National Cooperative for the Disposal of Radioactive Waste}
\newacronym{CUBIT}{CUBIT}{CUBIT Geometry and Mesh Generation Toolkit}
\newacronym{CSNF}{CSNF}{Commercial Spent Nuclear Fuel}
\newacronym{DSNF}{DSNF}{DOE Spent Nuclear Fuel}
\newacronym{HTGR}{HTGR}{High Temperature Gas Reactor}
\newacronym{TRISO}{TRISO}{Tristructural Isotropic}
\newacronym{MA}{MA}{Minor Actinide}
\newacronym{CEA}{CEA}{Commissariat a l'Energie Atomique et aux Energies Alternatives}
\newacronym{SKB}{SKB}{Svensk Karnbranslehantering AB}
\newacronym{SINDAG}{SINDA{\textbackslash}G}{Systems Improved Numerical Differencing Analyzer $\backslash$ Gaski}
%\newacronym{<++>}{<++>}{<++>}

\makeglossaries


%%%%%%%%%%%%%%%%%%%%%%%%%%%%%%%%%%%
\title{Thermal Calibration of an Analytical Repository Model using SINDA}
\author{Kathryn D.~Huff, Theodore H.~Bauer}

%% uncomment these next five only if using anstrans
\institute{Department of Nuclear Engineering \& Engineering Physics, University 
of Wisconsin, Madison, WI, 53706\\
Nuclear Engineering Division, Argonne National Laboratory, Argonne, IL, 60439}
\email{khuff@cae.wisc.edu \\
thbauer@anl.gov}
\usepackage{graphicx}
\usepackage{booktabs} % nice rules for tables
\usepackage{microtype} % if using PDF
\newcommand{\units}[1] {\:\text{#1}}%
\newcommand{\SN}{S$_N$}%{S$_\text{N}$}%{$S_N$}%


\date{January 27, 2012}
%%%%%%%%%%%%%%%%%%%%%%%%%%%%%%%%%%%
\begin{document}
%%%%%%%%%%%%%%%%%%%%%%%%%%%%%%%%%%%%%%%%%%%%%%%%%%%%%%%%%%%%%%%%%%%%%%%%%%%%%%%%
\section{Introduction}

This work proposes a simple calibrated thermal resistance model to improve the 
accuracy of a rapid homogenous medium analytical transient heat transfer 
solution model. The model to be calibrated, created by \gls{LLNL}, is a solution 
of superimposed line and point source solutions representing a generic geologic 
repository with a gridlike waste package layout in a homogeneous medium. The 
auxillary thermal resistance model improves the model's estimation of the peak 
temperature and peak time. The calibration of this model is undertaken as part 
of a benchmarking effort against a geometrically detailed generic thermal 
repository model based on \gls{SINDAG} software.

Calibrated an analytic thermal repository model from (\gls{LLNL}) model using a 
thermal resistance model and determined appropriate coefficients from 
\gls{SINDAG} calculations. 


%%%%%%%%%%%%%%%%%%%%%%%%%%%%%%%%%%%%%%%%%%%%%%%%%%%%%%%%%%%%%%%%%%%%%%%%%%%%%%%%
\subsection{Description of Work}

A benchmarking effort between the analytical \gls{LLNL} model and the 
\gls{SINDAG} \gls{ANL} model revealed a discrepancy between them. The time of 
peak heat arrived sooner and the value of the peak temperature was lower in the 
homogeneous medium \gls{LLNL} model than in the \gls{SINDAG} model. 

The SINDA model is based on a lumped parameter construction that models parts of 
the repository in detail. 

The LLNL model

Calibration was necessary.

Calibration occured for geologies of interest.  


%%%%%%%%%%%%%%%%%%%%%%%%%%%%%%%%%%%%%%%%%%%%%%%%%%%%%%%%%%%%%%%%%%%%%%%%%%%%%%%%
\section{Results and Analysis}

The calibration was an improvement on the benchmarking.

%%%%%%%%%%%%%%%%%%%%%%%%%%%%%%%%%%%%%%%%%%%%%%%%%%%%%%%%%%%%%%%%%%%%%%%%%%%%%%%%
\section{Conclusions}

We recommend that for this and other analytical models of this nature, the 
additional step will improve results near the area of interest.

%%%%%%%%%%%%%%%%%%%%%%%%%%%%%%%%%%%%%%%%%%%%%%%%%%%%%%%%%%%%%%%%%%%%%%%%%%%%%%%%
%\nocite{<+ +>}
\bibliographystyle{ans}
\bibliography{bibliography}
\end{document}


