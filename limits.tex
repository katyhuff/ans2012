% Waste is a problem
% Decisionmakers are contemplating many fuel cycle options
% Decisionmakers are contemplating many repository options
% Interfacing between FCO/SA campaign and UFD campaign


\begin{frame}[ctb!]
  \frametitle{Methods of Heat Transport in Various Geologies}
  \input{heat_tab.tex}
  Similar heat transport models can be used for all geologies, but are 
  differentiated by material parameters $(c_p, K, \rho)$, geometric parameters 
  (tunnel spacing, tunnel radius), and different geochemical constraints.
\end{frame}

\begin{frame}[ctb!]
  \frametitle{Heat Limits In Various Geologies}
  % Table
  Important heat limits in materials of the repository restrict loading designs 
  and capacity.  
  %        File: geos_tab.tex
%     Created: Thu Aug 04 11:00 AM 2011 C
% Last Change: Thu Aug 04 11:00 AM 2011 C
%
\begin{table}[h!]
  \centering
  \footnotesize{
  \begin{tabular}{|l|r|r|r|r|}
    \multicolumn{5}{c}{\textbf{Thermal Behavior of Various Concepts}}\\
    \hline
    Feature & Clay & Granite & Salt & Deep Borehole \\ 
    \hline
    Host Rock Limit $[^{\circ}C]$ & $\sim125$ & $\sim200$ & $\sim180$ & $>200$ \\ 
    Buffer Limit $[^{\circ}C]$ & 100 (Fo-Ca) & 100 (Fo-Ca) & 180 & 100 (Fo-Ca)\\ 
    Conductivity $[\frac{W}{m{\cdot}K}]$ & $1-2$ & $2-4$ & $\sim4$  & $2-4$ \\ 
    Diffusivity $[\frac{m^2}{s}]$ & $1-6\times10^{-7}$ & $1\times10^{-6}$ & $1-2\times10^{-6}$  & $1\times10^{-6}$ \\ 
    Coalesence & yes & no & yes & no \\ 
    \hline
  \end{tabular}
  \caption{Reference values for thermal limits and behaviors in various 
  candidate repository geologies.}
  }
  \label{tab:geos_tab}
\end{table}
%  \cite{stober_hydraulic_2006} 

\end{frame}

 \begin{frame}
   \frametitle{Impact of Repository Geologies}
   \begin{figure}[h!]
     \begin{center}
       \includegraphics[height=.7\textheight]{llnlGeos.eps}
     \end{center}
     \caption{LLNL has found that the higher heat limit, alcove geometry, and 
     high conductivity in salt allows for earlier loading times.}
     \label{fig:llnlGeos}
   \end{figure}
\end{frame}

% heat based capacity 
\begin{frame}[ctb!]
  \frametitle{Impact of Repository Designs}
   \input{footprint_tab.tex}
 \end{frame}

