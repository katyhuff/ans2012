

\begin{frame}[ctb!]
  \frametitle{Heat Limits In Geology}
  % table?
  Important heat limits in materials of the repository restrict loading designs 
  and capacity.
  \input{heat_tab.tex}
\end{frame}

% heat based capacity 
\begin{frame}[ctb!]
  \frametitle{Impact of Repository Designs}
  % lines, points, infinite lines, footprints.
   \input{footprint_tab.tex}
   \begin{figure}[h!]
     \begin{center}
       \includegraphics[height=.5\textheight]{llnlGeos.eps}
     \end{center}
     \caption{LLNL has found that the higher heat limit, alcove geometry, and 
     high conductivity in salt allows for earlier loading times.}
     \label{fig:llnlGeos}
   \end{figure}
\end{frame}

\begin{frame}[ctb!]
  \frametitle{Detailed Techniques}
  % 2d,3d,finite diffs, etc.
  \input{heat_tab.tex}
  Similar heat transport models can be used for all geologies, but are 
  differentiated by material parameters $(c_p, K, \rho)$ and different 
  thermal constraints.
\end{frame}
