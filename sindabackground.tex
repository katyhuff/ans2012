

% SINDA 
\begin{frame}[ctb!]
  \frametitle{Numerical Model : SINDA{\textbackslash}G Technique}
  This techinique was originally created for optimal waste loading analysis of 
  YMR by the UFD team at Argonne national lab. It  uses the 
  SINDA{\textbackslash}G heat transport framework which employs a geometrically 
  precise numerical lumped parameter model. 
  \begin{figure}[h!]
    \begin{center}
      \includegraphics[height=.5\textheight]{sindageom.eps}
    \end{center}
    \caption{The geometry of the 2D thermal model can be adjusted by altering 
    tunnel diameter, tunnel spacing, and the vertical distance below the 
    surface.}
    \label{fig:sindageom}
  \end{figure}
\end{frame}


\begin{frame}[ctb!]
  \frametitle{Numerical Model : Lumped Parameter Technique}
  % resistor diagram
  \begin{figure}[h!]
    \begin{center}
      \includegraphics[width=0.9\textwidth]{lumpedParam.eps}
    \end{center}
    \caption{The lumped parameter analogy used for heat transfer can be applied 
    as a one dimensional approximation to the disposal system concept. }
    \label{fig:lumpedParam}
  \end{figure}
\end{frame}

\begin{frame}[ctb!]
\frametitle{Numerical Model : Calculation Method}
The SINDA{\textbackslash}G  lumped capacitance tool solves a thermal circuit, for which 
conducting nodes may be of four types corresponding to the four modes of heat 
transfer. Nodes are connected by conduction, convection, radiation, and mass 
flow heat transfer links. In the SINDA{\textbackslash}G engine, these are represented by
\footnotesize{
\begin{align}
  R_{rad}  &= \frac{1}{\sigma F_{ij}A\left[ T_i + T_A + T_j + T_A 
  \right]\left[(T_i+T_A)^2+(T_j+T_A)^2\right]}\nonumber\\
  R_{cond} &= \frac{L}{K_{th} A}\mbox{, }R_{conv} = \frac{1}{h A}\mbox{, and 
  }R_{mf} = \frac{1}{\dot{m}c_p}
  \intertext{where}
  K_{th}&= ~~\mbox{thermal conductivity}[W\cdot m^{-1}\cdot K^{-1}]\nonumber\\
  A&= ~~\mbox{area} [m^2]\nonumber\\
  c_p&=~~\mbox{specific heat capacity} [J\cdot K^{-1}]\nonumber  \\
  h&= ~~\mbox{heat transfer coefficient}[W\cdot m^{-1} \cdot K^{-1}]\nonumber \\
  \dot{m}&= ~~\mbox{mass transfer rate}[kg\cdot s^{-1}]\nonumber \\
  T_i&= ~~\mbox{lump temperature} [^{\circ}C] \nonumber\\
  T_A&= ~~\mbox{absolute temperature} [^{\circ}C] \nonumber\\
  F_{ij}&= ~~\mbox{radiation interchange factor} [-] .\nonumber
\end{align}
}
\end{frame}


\begin{frame}[ctb!]
  \frametitle{Numerical Model : SINDA{\textbackslash}G Geometries}

Two SINDA{\textbackslash}G model geometries have been used in this benchmark.  
\begin{itemize}
  \item{\textbf{Single Drift}} In the single drift geometry, there is a distant fixed boundary condition and one waste tunnel is modeled with a continuous, cylindrical heat source of infinite length. The linear heat source in $[\frac{W}{m}]$ is modeled as if it is spread azimuthally over the surface of the drift tunnel.  
  \item{\textbf{Multiple Drift}} As llustrated in Figure \ref{fig:sindageom}, an infinite array of identical single-drift heat sources is modeled, by assuming one-half of a storage tunnel with a reflective boundary condition at a vertical plane midway between drifts. 
\end{itemize}
\end{frame}
