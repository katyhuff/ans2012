

% SINDA 
\subsection{Background: Numerical SINDA{\textbackslash}G Model}

\begin{frame}[ctb!]
  \frametitle{Background: Numerical SINDA{\textbackslash}G Model}
  A model created by the UFD team at Argonne national lab using the 
  SINDA{\textbackslash}G heat transport framework employs a lumped parameter 
  model and an optimization loop to arrive at a minimal drift spacing for a 
  given waste stream in agreement with user input thermal limits. 
  \begin{figure}[h!]
    \begin{center}
      \includegraphics[height=.5\textheight]{sindageom.eps}
    \end{center}
    \caption{Two adjustable geometric dimensions of the ANL model.} 
    \label{fig:sindageom}
  \end{figure}
\end{frame}


\begin{frame}[ctb!]
  \frametitle{Lumped Parameter Technique}
  % resistor diagram
  \begin{figure}[h!]
    \begin{center}
      \includegraphics[width=0.9\textwidth]{lumpedParam.eps}
    \end{center}
    \caption{The lumped parameter analogy used for heat transfer can be applied 
    as a one dimensional approximation to the disposal system concept. }
    \label{fig:lumpedParam}
  \end{figure}
\end{frame}

%%%%%%%%%%%%%%%%%%%%%%%%%%%%%%%%%%%%%%%%%%%%%%%%%%%%%%%%%%%%%%%%%%%%%%%%%%%%%%%%

%% The ANL model

\begin{frame}[ctb!]
  \frametitle{Background: Numerical SINDA{\textbackslash}G Model}

  The numeric heat transport model created by the \gls{UFD} team at \gls{ANL} 
  using the \gls{SINDAG} heat transport framework employs detailed finite-difference numeric  
  models describing two distinct geometric arrangements: 

  \begin{enum}
  \item a single storage drift 
  \item an infinite array of identical, uniformly spaced  storage drifts.  
  \end{enum}
  
  For a given waste stream, tunnel radius, and geologic parameters (i.e. thermal conductivity, density, and 
  specific heat capacity), the model is able to compute the temperature field 
  surrounding the tunnel wall and beyond.  
\end{frame}

\begin{frame}[ctb!]
\frametitle{Calculation Method}

The \gls{SINDAG} calculation engine uses a lumped parameter numeric model.
Originally created for optimal waste loading analysis of the \gls{YMR}, the 
model for an array of drifts is geometrically adjustable,  as illustrated in 
Figure \ref{fig:sindageom}. 

\begin{figure}[htbp!]
  \begin{center}
    \includegraphics[width=0.4\textwidth]{./sindageom.eps}
  \end{center}
  \caption{The geometry of the 2D thermal model can be adjusted by altering 
  tunnel diameter, tunnel spacing, and the vertical distance below the surface.}
  \label{fig:sindageom}
\end{figure}
\end{frame}

\begin{frame}[ctb!]
\frametitle{Calculation Method}
The \gls{SINDAG} lumped capacitance tool solves a thermal circuit, for which 
conducting nodes may be of four types corresponding to the four modes of heat 
transfer. Nodes are connected by conduction, convection, radiation, and mass 
flow heat transfer links. In the \gls{SINDAG} engine, these are represented by

\begin{align}
  R_{rad}  &= \frac{1}{\sigma F_{ij}A\left[ T_i + T_A + T_j + T_A 
  \right]\left[(T_i+T_A)^2+(T_j+T_A)^2\right]}\nonumber\\
  R_{cond} &= \frac{L}{K_{th} A}\mbox{, }R_{conv} = \frac{1}{h A}\mbox{, and 
  }R_{mf} = \frac{1}{\dot{m}c_p}
  \intertext{where}
  K_{th}&= ~~\mbox{thermal conductivity}[W\cdot m^{-1}\cdot K^{-1}]\nonumber\\
  A&= ~~\mbox{area} [m^2]\nonumber\\
  c_p&=~~\mbox{specific heat capacity} [J\cdot K^{-1}]\nonumber  \\
  h&= ~~\mbox{heat transfer coefficient}[W\cdot m^{-1} \cdot K^{-1}]\nonumber \\
  \dot{m}&= ~~\mbox{mass transfer rate}[kg\cdot s^{-1}]\nonumber \\
  T_i&= ~~\mbox{lump temperature} [^{\circ}C] \nonumber\\
  T_A&= ~~\mbox{absolute temperature} [^{\circ}C] \nonumber\\
  F_{ij}&= ~~\mbox{radiation interchange factor} [-] .\nonumber
\end{align}
\end{frame}


\begin{frame}[ctb!]
  \frametitle{SINDA{\textbackslash}G Geometries}

Two \gls{SINDAG} model geometries have been used in this benchmark.  

\textbf{Single Drift}
In the single drift geometry, there is a distant fixed boundary condition and 
one waste tunnel is modeled with a continuous, cylindrical heat source of 
infinite length. The linear heat source in $[\frac{W}{m}]$ is modeled as if it 
is spread azimuthally over the surface of the drift tunnel. 

\textbf{Multiple Drift}
As llustrated in Figure \ref{fig:sindageom}, an infinite array of identical single-drift heat sources is modeled,
by assuming one-half of a storage tunnel with a reflective boundary condition at a vertical
plane midway between drifts. 

\end{frame}
