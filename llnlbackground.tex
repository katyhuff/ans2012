% LLNL

\begin{frame}
  \frametitle{Analytical Model : Geometry}
  \begin{minipage}{0.3\textwidth}
    \begin{figure}[h!]
      \includegraphics[width=\textwidth]{boreholes.eps}
    \end{figure}
    \begin{figure}[h!]
      \includegraphics[width=\textwidth]{vertical.eps}
    \end{figure}
  \end{minipage}
  \hspace{0.01cm}
  \begin{minipage}{0.3\textwidth}
    \begin{figure}[h!]
      \includegraphics[width=\textwidth]{horizontal.eps}
    \end{figure}
    \begin{figure}[h!]
      \includegraphics[width=\textwidth]{alcoves.eps}
    \end{figure}
  \end{minipage}
  \hspace{0.01cm}\large{$=$}\hspace{0.01cm}
  \begin{minipage}{0.3\textwidth}
    \begin{figure}[b]
      \includegraphics[width=\textwidth]{fullGrid.eps}
    \end{figure}
  \end{minipage}
\end{frame}

\begin{frame}
  \frametitle{Analytical Model : Geometry}
  \begin{figure}[h!]
    \begin{center}
      \includegraphics[width=0.7\textwidth]{llnlConcept.eps}
    \end{center}
    \caption{Vertical, horizontal, alcove, and borehole emplacement layouts can 
    be represented by a line of point sources and adjacent line sources 
    \cite{sutton_investigations_2011}.}
    \label{fig:llnl}
  \end{figure}
\end{frame}

\begin{frame}
  \frametitle{Analytical Model : Solution Strategy}
    LLNL's model is a MathCAD solution of the transient homogeneous 
    conduction equation,
    
    \begin{align}
      \nabla^2T  = \frac{1}{\alpha}\frac{\partial T}{\partial t},
      \label{condGl}
    \end{align}
    
    in which superimposed point and line source solutions approximate the repository 
    layout.
\end{frame}

\begin{frame}[ctb!]
\frametitle{Analytical Model Background}
The analytic model, created at LLNL for the UFD campaign seeks to 
inform heat limited waste capacity calculations for each lithology, for many 
waste package loading densities, and for many fuel cycle options 
\cite{hardin_generic_2011, sutton_investigations_2011, 
greenberg_application_2012}. It employs an analytic model from Carslaw and 
Jaeger and is implemented in MathCAD \cite{carslaw_conduction_1959, 
ptc_mathcad_2010}.  The integral solver in the MathCAD toolset is the primary 
calculation engine for the analytic MathCAD thermal model, which relies on 
superposition of integral solutions.  
\end{frame}

\begin{frame}[ctb!]
\frametitle{Calculation Method}
The model consists of two conceptual regions, an external region representing 
the host rock and an internal region representing the waste form, package, and 
buffer Engineered Barrier System within the disposal tunnel wall. 
\begin{itemize}
  \item Since the thermal mass of the EBS is small in comparison to the thermal mass of the host rock, the internal region may be treated as quasi-steady state. 
  \item The transient state of the temperature at the calculation radius is found with a convolution of the transient external solution with the steady state internal solution.  
  \item The process is then iterated with a one year resolution in order to arrive at a temperature evolution over the lifetime of the repository. 
\end{itemize}
\end{frame}


\begin{frame}[ctb!]
\frametitle{Geometry}
\begin{minipage}{0.3\textwidth}
\begin{figure}[h!]
  \begin{center}
    \includegraphics[width=0.5\textwidth]{llnlConcept.eps}
  \end{center}
  \caption{The central package is represented by a finite line source, adjacent 
  packages in the central drift are represented as points, and adjacent disposal 
  tunnels are represented as infinite lines.
  \cite{sutton_investigations_2011}.}
  \label{fig:llnl}
\end{figure}
\end{minipage}
TEST
\end{frame}

\begin{frame}[ctb!]
  \frametitle{Geometry}
The geometric layout of the analytic LLNL model in Figure \ref{fig:llnl} 
shows  that the central package is represented by the finite line solution
\begin{align}
  T_{line}(t,x,y,z) &= \frac{1}{8\pi K_{th}} 
  \bigintsss_0^t\!\frac{q_L(t')}{t-t'}e^{ \frac{-\left(x^2 + z^2\right)}{4\alpha 
  (t-t')} }\nonumber\\ &\cdot\left[ \erf{\left[ \frac{1}{2} \frac{\left( y + 
  \frac{L}{2} \right)}{\sqrt{\alpha(t-t')}}  \right]} - \erf{\left[ \frac{1}{2} 
  \frac{\left( y - \frac{L}{2} \right)}{\sqrt{\alpha(t-t')}}  \right]} 
  \right]\,\mathrm{dt'},
  \label{line}
  \intertext{adjacent packages within the central tunnel are represented by the 
  point source solution }
  T_{point}(t,r) &= 
  \frac{1}{8K_{th}\sqrt{\alpha}\pi^{\frac{3}{2}}}\bigintsss_0^{-t}\!\frac{q(t')}{(t-t')^{\frac{3}{2}}}e^{\frac{-r^2}{4\alpha(t-t')}}\,\mathrm{dt'},
  \label{point}
  \intertext{and adjacent disposal tunnels are represented by infinite line 
  source solutions}
  T_{\infty line}(t,x,z) &= \frac{1}{4\pi K_{th}} 
  \bigintsss_0^t\!\frac{q_L(t')}{t-t'}e^{ \frac{-\left(x^2 + z^2\right)}{4\alpha 
  (t-t')} }
  \intertext{in infinite homogeneous media, where}
  \label{infline}
  \alpha &= ~~\mbox{thermal diffusivity } [m^2\cdot s^{-1}]\nonumber\\
  q(t) &= ~~\mbox{point heat source} [W]\nonumber\\
  \intertext{and}
  q_L(t) &= ~~\mbox{linear heat source} [W\cdot m^{-1}]\nonumber
\end{align}
Superimposed point and line source solutions allow for a notion of the 
repository layout to be modeled in the host rock.
\end{frame}


